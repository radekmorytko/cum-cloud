\RequirePackage{fix-cm}
%
%\documentclass{svjour3}                     % onecolumn (standard format)
%\documentclass[smallcondensed]{svjour3}     % onecolumn (ditto)
%\documentclass[smallextended]{svjour3}       % onecolumn (second format)
\documentclass[twocolumn]{svjour3}          % twocolumn
%
\usepackage[utf8]{inputenc}
\smartqed  % flush right qed marks, e.g. at end of proof
%
\usepackage{graphicx}
\usepackage{amssymb}

%
% \usepackage{mathptmx}      % use Times fonts if available on your TeX system
%
% insert here the call for the packages your document requires
%\usepackage{latexsym}
% etc.
%
% please place your own definitions here and don't use \def but
% \newcommand{}{}
%
% Insert the name of "your journal" with
\journalname{Computer Science Journal -- AGH University of Science and Technology}
%
\begin{document}

\title{Cloud-SAP: Self-adaptive platform for scaling users' application deployed in a federation of cloud computing environments}

%\subtitle{Do you have a subtitle?\\ If so, write it here}

%\titlerunning{Short form of title}        % if too long for running head
\titlerunning{Cloud-SAP: Self-adaptive platform for scaling users' application in a federation of cloud computing environments}

\author{Dariusz Chrząścik          \and
        Radosław D. Morytko
}

%\authorrunning{Short form of author list} % if too long for running head

\institute{
  D. Chrząścik \\
  Chief Technology Officer at Software Mind SA \\
  Enthusiast of gorgeous cars and fast women \\
  Amateur poet, screenwriter and novelist\at
  \email{dariusz@chrzascik.com}           %  \\
  \and
  R.D. Morytko \\
  Software Developer and Board Member at Dropsport \\
  Devoted admirer of Gothic cathedrals \\
  Amateur guitar player and zumba dancer\at
  \email{radoslaw@morytko.pl}
}

\date{Received: 11 Sep 2013 / Accepted: 15 Sep 2013}

\maketitle

\begin{abstract}
In order to retain the lead among its competitors, a software company needs to create a distributed system which is spanned across different geographical locations, vulnerable to sudden variations in demand and has remarkable QoS requirements. Following current trends it chooses Cloud as a deployment platform. Unfortunately, existing cloud providers do not have tools and mechanisms that would enable dynamic load distribution among different data centers to meet aforementioned requirements.
In this article we want to outline the architecture of a self-adaptive platform (\emph{Cloud-SAP}) which facilitates scalable provisioning users' applications and fulfills QoS needs under variable conditions. Our solution can be considered a multi-layered environment for users' services as it applies the notion of an autonomic system to its every layer on a which auto scaling can be executed -- application, container, service and cloud. To ensure meeting QoS requirements at the cloud level we use the recent concept of a utility-oriented federation of cloud environments (InterCloud).
We present and compare current cloud solutions with the emphasis on their scaling capabilities. Finally, we demonstrate our preliminary results of conducted evaluation studies on the CloudSim toolkit. 

\keywords{cloud computing\and application scaling \and intercloud \and hybrid configuration}
% \PACS{PACS code1 \and PACS code2 \and more}
% \subclass{MSC code1 \and MSC code2 \and more}
\end{abstract}

\section{Introduction}
\label{intro}
Cloud computing, a relatively new concept which denotes offering computing utilities on demand in a pay-as-you-go manner like any other services available in today's society, has already started the process of transforming IT industry, making an impact on the way the software/hardware is produced and perceived. Because of its widespread presence in the IT community, this model makes computing resources more and more attractive for broader audience, especially in diverse branches of business, e.g. pharmaceutical or construction.

There are a few noticeable things which can be considered innovative in the way the services are provided. Firstly, clients of such services have no knowledge of the physical location of resources they use. Secondly, they are under illusions of infinite resources that are available. What is more, they pay providers only when they really use their services -- this means they do not need to invest heavily in building and maintaining hardware infrastructure. Last but not least, in order to ease dynamic resource management and manipulation, cloud providers use various virtualization technologies which operate on different levels, e.g. server, network or storage.

\subsection{Service Models}
Cloud computing classified the delivered services to three categories: Infrastructure as a Service (IaaS), Platform as a Service (PaaS) and Software as a Service (SaaS). They provide the consumer with infrastructure, deployment platform (application ecosystem) and software respectively. The just defined terms are commonly referred to in Cloud community as \emph{service models}.

\subsection{Deployment models and cloud federation}

\subsection{Application scaling}


\section{Cloud providers -- comparison}

\begin{table*}[!ht]
  \renewcommand{\arraystretch}{2}
\begin{tabular}{ l | c | c | c c }
  \hline 
  \multicolumn{1}{l}{} & \multicolumn{1}{l}{\textbf{Horizontal scaling}} & \multicolumn{1}{l}{\textbf{Vertical scaling}} & \textbf{Application / resource tuning} \\ \hline

  \multicolumn{1}{l}{\textbf{Infrastructure provider}} & \multicolumn{4}{ l }{} \\ \hline

Carina & \checkmark & $\times$ & $\times$ & \\ \hline

OneFlow & \checkmark & $\times$ & $\times$ & \\ \hline

AWS EC2 & \checkmark & $\times$ & $\times$ & \\ \hline

\multicolumn{1}{l}{\textbf{Platform provider}} & \multicolumn{4}{ l }{} \\ \hline

CloudFoundry & $\times$ & $\times$ & $\times$  &\\ \hline

OpenShift & \checkmark & $\times$ & $\times$  &\\ \hline

AppEngine & \checkmark & $\times$ & $\times$  &\\ \hline

Azure & \checkmark & $\times$ & $\times$  &\\ \hline

Heroku & \checkmark & $\times$ & $\times$ & \\ \hline
\end{tabular}

\caption{Comparision of cloud providers scaling capabilites}
\label{tab:cloud-providers-scaling}

\end{table*}

\section{Architecture of SAP-Cloud}

\section{Preliminary test results}

\section{Summary and conclusion}

\section{Section title}
\label{sec:1}
Text with citations \cite{RefB} and \cite{RefJ}.
\subsection{Subsection title}
\label{sec:2}
as required. Don't forget to give each section
and subsection a unique label (see Sect.~\ref{sec:1}).
\paragraph{Paragraph headings} Use paragraph headings as needed.
\begin{equation}
a^2+b^2=c^2
\end{equation}

% For one-column wide figures use
%\begin{figure}
% Use the relevant command to insert your figure file.
% For example, with the graphicx package use
%  \includegraphics{example.eps}
% figure caption is below the figure
%\caption{Please write your figure caption here}
%\label{fig:1}       % Give a unique label
%\end{figure}
%
% For two-column wide figures use
%\begin{figure*}
% Use the relevant command to insert your figure file.
% For example, with the graphicx package use
%  \includegraphics[width=0.75\textwidth]{example.eps}
% figure caption is below the figure
%\caption{Please write your figure caption here}
%\label{fig:2}       % Give a unique label
%\end{figure*}
%
% For tables use
%\begin{table}
% table caption is above the table
%\caption{Please write your table caption here}
%\label{tab:1}       % Give a unique label
% For LaTeX tables use
%\begin{tabular}{lll}
%\hline\noalign{\smallskip}
%first & second & third  \\
%\noalign{\smallskip}\hline\noalign{\smallskip}
%number & number & number \\
%number & number & number \\
%\noalign{\smallskip}\hline
%\end{tabular}
%\end{table}


%\begin{acknowledgements}
%If you'd like to thank anyone, place your comments here
%and remove the percent signs.
%\end{acknowledgements}

% BibTeX users please use one of
%\bibliographystyle{spbasic}      % basic style, author-year citations
%\bibliographystyle{spmpsci}      % mathematics and physical sciences
%\bibliographystyle{spphys}       % APS-like style for physics
%\bibliography{}   % name your BibTeX data base

% Non-BibTeX users please use
\begin{thebibliography}{}
%
% and use \bibitem to create references. Consult the Instructions
% for authors for reference list style.
%
\bibitem{RefJ}
% Format for Journal Reference
Author, Article title, Journal, Volume, page numbers (year)
% Format for books
\bibitem{RefB}
Author, Book title, page numbers. Publisher, place (year)
% etc
\end{thebibliography}

\end{document}


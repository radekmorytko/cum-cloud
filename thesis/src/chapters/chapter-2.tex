\chapter{Scaling applications}

\section{Introduction}

The reason why this dissertation treats about scaling application is the fact that it is widely agreed measure for improving application performance, consequently increasing offered Quality-of-Service. Enriching system with capability to scale entails avoiding additional costs that are related to coping with excessive traffic. In some cases, these costs may be caused by not handling extra traffic at all or may involve aspects such as: increased response time, processing overhead, space, memory, or money \cite{Bo00}. 

While scalability is a widely used term, it still lacks a clear and concise definition. Over the time, there were a few attempts to define it, yet not all of them were claimed as successful \cite{Hi90} \cite{DuRoWi06}. Hence, it is necessary to clarify this term before going into further discussion. Instinctively, scalability is perceived as ability of a system to accommodate an increasing number of elements or objects to process. In particular, we can define different aspects of scalability \cite{Bo00}:
\begin{itemize}
	\item Load scalability,
	\item Space scalability,
	\item Space-time scalability,
	\item Structural scalability,
\end{itemize}

\cite{EaZaLa89}:
\begin{equation}
	
\end{equation}

With all that said, there is no silver bullet - not matter what underlying mechanism provider decides to use, the application developer is still responsible for creating an application with scaling in-mind. The idea behind this statement has been already proven in 1967 by Amdahl law, which in short states that sequential component of parallel algorithms impacts efficiency for a sufficiently large number processors \cite{Am67}.


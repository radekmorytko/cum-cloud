\chapter{Summary}

\chapterintro{This chapter provides the summery of the dissertation and gives information about future work and research directions.}

\section{Conclusions}
In this dissertation we proposed an architecture of a platform that ensures satisfying Quality of Service requirements of users' applications deployed in a cloud computing environment. The key attribute of the proposed system is holistic approach to self-adaptability achieved by insightful understanding of platform-as-a-service model. As a consequence, designed architecture manages variety of resources that users' applications incorporate leading to a system that is capable of vertical, horizontal and most importantly cloud-federation aware scaling. Additionally, we implemented a minimal viable product which conforms to the devised architecture and made its evaluation by performing tests which compare it to some of the currently available solutions.

% tutaj mozna by wymienic krotko rozwiazania czyli ze autonomiczny, vertykalny na konetenerach - i chyba tyle
Cloud-SAP derives ability to conduct self-monitoring, analysing, planning and executing from mature and widely respected model of an autonomic system. 

The architecture proposal and its implementation clearly indicates that the goals of the dissertation stated in the first chapter have been achieved. Since the implementation is merely a proof-of-concept, it cannot be seen as a complete or reference solution, though. Irrespective of the simplicity of the employed prediction, planning or resource mapping algorithms, promising test results were obtained. However, in order to get better results, more sophisticated algorithms and methods should be implemented in place of them.

All tests of the platform were not simulated on any software, but conducted on the real hardware forming a home-made data center. Tests were chosen and design so as to reflect at the highest possible rate real business use cases. Their results show considerable potential of the proposed solution.


TODO co z OpenVZ--KVM?, hmm to jest wpisane pod vertical scaling, mozna by ewentualnie jakos wydzielic te poziomy skalowania i je opisac

\section{Future work}
As the provided implementation cannot be regarded as a fully fledged one, there is a need to implement a system with the advanced algorithms regarding actions in a control loop of autonomic managers such as:
\begin{itemize}
  \item prediction
  \item prioritization
  \item application self-tuning
\end{itemize}
This will enable to conduct more insightful tests of the architecture and prove its usefulness.

Having more mature platform implemented, next step should be to engage communities working on Open Source cloud platform such as OpenNebula, OpenStack or Eucalyptus. It is expected that such cooperation broaden pool of supported cloud providers and thus truly enable diversified federation of clouds.

% moze cos o wyzwaniach z tym zwiazanych, ze trzeba sie dogadac, ustalic wspolny model, format

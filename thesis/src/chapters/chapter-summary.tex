\chapter{Summary}

\chapterintro{This chapter provides the summary of the dissertation and gives information about future work and research directions.}

\section{Conclusions}
In this dissertation we proposed an architecture of a platform that ensures satisfying Quality of Service requirements of users' applications deployed in a cloud computing environment. The key attribute of the proposed system is holistic approach to a self-adaptability achieved by an insightful understanding of a platform-as-a-service model. As a consequence, designed architecture manages variety of resources that users' applications incorporate leading to a system that is capable of vertical, horizontal and most importantly cloud-federation aware scaling. Additionally, we implemented a minimal viable product which conforms to the devised architecture and made its evaluation by performing tests which compare it to some of the currently available solutions.

The architecture proposal and its implementation clearly indicates that the goals of the dissertation stated in the first chapter have been achieved. Since the implementation is merely a proof-of-concept, it cannot be seen as a complete or reference solution, though. Irrespective of the simplicity of the employed prediction, planning or resource mapping algorithms, promising test results were obtained. However, in order to get better results, more sophisticated algorithms and methods should be implemented in place of them.

All tests of the platform were not simulated on any software, but conducted on the real hardware forming a home-made data center. Tests were chosen so as to reflect at the highest possible rate real business use cases. Their results show considerable potential of the proposed solution. Nevertheless, we did not manage to evaluate implementation in a real-world scenario that entails using enterprise-level application such as the ones based on Java Enterprise Edition.

\section{Future work}
As the provided implementation cannot be regarded as a fully fledged one, there is a need to implement a system with the advanced algorithms regarding actions in a control loop of autonomic managers such as:
\begin{inparaenum}[1)]
  \item prediction
  \item service prioritization
  \item application self-tuning
\end{inparaenum}
This will enable to conduct more insightful tests of the architecture and prove its usefulness.

Beside this, the implementation should became a business-ready product instead of being simply a proof-of-concept. What should be done to enact it is to:
\begin{itemize}
 \item enrich Cloud-SAP architecture with a number of financial aspects such as service maintenance cost, policy and customer oriented billings (i.e. fee should take into account customer-specified policy or customer himself).
 \item evaluate solution against real-world benchmarks such as SPECweb96, SPECweb99 or TPC-C
\end{itemize}
Due to Cloud-SAP flexibility, above-mentioned features should be easily and seamlessly incorporated in proposed model.


Having more mature platform implemented and tested, next step should be to engage communities working on Open Source cloud platform such as OpenNebula, OpenStack or Eucalyptus. It is expected that such cooperation broaden pool of supported cloud providers and thus truly enable diversified federation of clouds. However, for cloud providers to cooperate, there is a need for discussion and consideration regarding aspects such as authentication, authorization, resource
 access, resource discovery, to name a few.


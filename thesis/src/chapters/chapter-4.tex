\chapter{Interoperability of clouds}

\chapterintro{This chapter introduces the notion of a hybrid cloud and explains its role in cloud computing industry. On top of this deployment model, the concept of InterCloud is presented and elaborated with the emphasis on its application in ensuring scalability of users' services.}

\section{Introduction}
From the perspective of a user of PaaS services it is vital that they are able to deploy seamlessly their applications using libraries, tools and services supported by the cloud provider\cite{MeGr11}. Judging by such factors as the popularity of Heroku -- currently one of the most popular PaaS providers which does not offer more advanced features which would enable management of the infrastructure underpinning the deployment platform, the fact that Microsoft added auto-scaling to its Azure platform as late as in June 2013, it is perfectly possible most PaaS users are satisfied with the current offers of their providers and do need another, more sophisticated functionalities. However, there are more complex applications and systems whose requirements regarding technology stack, availability and scalability are considerably more demanding. For such services there ought to be designed slightly specialized features that would require cooperation among different cloud providers.

\section{Hybrid cloud}
One can imagine scenarios in which customers of cloud services know their applications are vulnerable to sudden variations in demand and their responsiveness must be kept at the same level all the time. In such cases, they want them to scale dynamically according to current load or other predefined or manually specified metrics. What is more, in order to ensure high availability of their services, customers do not want to confine themselves to only one provider -- in the best scenario they want their applications (or their logical parts, such as persistence layer) to be spanned across different providers and be able to cooperate with one another at the same time. This leads to the concept of a \emph{hybrid cloud}\cite{MeGr11} -- the case in which the cloud infrastructure is a composition of two or more distinct infrastructures which are unique entities, but there are technological means that make it possible to port data and applications among them.

\subsection{Usage in industry}

\section{Federation of clouds -- InterCloud}

\subsection{Usage in industry}


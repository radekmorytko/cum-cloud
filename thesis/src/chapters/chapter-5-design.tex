\chapter{Design of Cloud-SAP}

\chapterintro{ This chapter introduces the high-level design of Cloud-SAP, highlighting its core concepts and indicating possible implementation ideas.
}

\section{Requirements}
One can notice that elements that yields a solution for a problem stated in the first chapter, which is ensuring that users' application provide appropriate Quality-of-Service for its customers in a most-cost effective manner, were gradually introduced in previous chapters:

\begin{itemize}
	\item \emph{scalability} - ability to improve application performance by enriching resources
	\item \emph{adaptivity} - ability to adapt (i.e. scale) appropriately to a current usage pattern
	\item \emph{inter-cloud awareness} - ability to compose an application deployment using different cloud providers; cooperation with different cloud provider to supply application with extra resources while performing application scaling
\end{itemize}

Next section states the general overview of the proposed solution, while the consecutive sections presents more detailed discussion of its elements and finally the last section summarises the design choices in a context of system requirements.
	
\section{High-level design}
\subsection{Overview}

\begin{figure}[!ht]
  \begin{center}
    \includegraphics[width=0.8\textwidth]{chapter-5/hld-overview}
  \end{center}
  \caption{Cloud-SAP high level overview}
  \label{design:hld-overview}
\end{figure}

Diagram \ref{design:hld-overview} illustrates exemplary Cloud-SAP deployment, among depicted components we can distinguish two main parts of proposed solution: auto-scaling subsystem and inter-cloud broker. Auto-scaling subsystem is responsible for scaling users' application, taking into account different scalability perspectives, handled by a different layers:
\begin{itemize}
	\item application platform layer: application platform tuning
	\item container layer: vertical scaling
	\item stack layer: horizontal scaling
	\item inter-cloud layer: scaling out across different cloud providers
\end{itemize}
The last mentioned layer - inter-cloud - cooperates with different cloud providers, hence, it requires additional work to be done. In fact, these extra tasks are delegated to a inter-cloud broker, responsible for a dialog in a cloud federation.

\begin{figure}[!ht]
  \begin{center}
    \includegraphics[width=0.5\textwidth]{chapter-5/sap-layers}
  \end{center}
  \caption{Layered structure of Cloud-SAP}
  \label{design:csap-layers}
\end{figure}

\subsection{Auto-scaling subsystem}

\subsubsection{Autonomic components}
Each layer of the Cloud-SAP (shown in figure \ref{design:csap-layers}) must be characterised by an ability to adapt to a given application usage. One of the first models that ensured system adaptivity is an autonomic component, concept based on a feed-back loop, initially proposed by IBM \cite{IBM06}. Figure \ref{design:autonomic-component} depicts that architecture. This observation is a foundation of proposed architecture - adaptivity is achieved by enriching each layer with an elasticity controller that is in fact an autonomic component.

\begin{figure}[!ht]
  \begin{center}
    \includegraphics{chapter-5/autonomic-component}
  \end{center}
  \caption{autonomic component}
  \label{design:autonomic-component}
\end{figure}

Due to the fact that designed platform operates on multiple layers, we can extend idea of an autonomic component by using concept of a multi-hierarchical autonomic system \cite{LiWoZh05}. Figure \ref{design:hierarchical-autonomic-system} illustrates that hierarchy, were each level represents a different perspective on an application scaling. First three hierarchy levels (application tuning, container, stack) are centralised and controlled by a single elasticity controller at each level, while the last inter-cloud level is a decentralized one - each cloud instance is fully independent.

\begin{figure}[!ht]
  \begin{center}
    \includegraphics[width=0.8\textwidth]{chapter-5/hierarchical-autonomic-system}
  \end{center}
  \caption{Cloud-SAP as an hierarchical autonomic system}
  \label{design:hierarchical-autonomic-system}
\end{figure}

Considering hierarchy of our architecture, each level manages an underlying autonomic component, while being managed by an upper layer at the very same time.

\subsubsection{Monitoring}
Monitoring module of each elasticity controller aims to gather data from \textit{Sensors} of a managed resource. While design of Cloud-SAP doesn't have any specific requirements regarding underlying monitoring mechanism, there are a few aspects that should be taken into consideration. Table \ref{tab:monitoring-module-issues-summary} summarises main points.

\begin{table}[!htbp]
\begin{tabularx}{\textwidth}[]{ X X X }
\specialrule{.1em}{.05em}{.05em}
 & \textbf{Advantages} & \textbf{Disadvantages} \\ \specialrule{.1em}{.05em}{.05em} 

 
\textbf{data exchange model} & & \\ \specialrule{.1em}{.05em}{.05em}
push based &
-- notification driven: all changes are reflected immediately in the system
&
-- requires additional agent at monitored component
\\ \hline
pull based & 
-- more calculations are being done at monitoring module side (doesn't involve additional load at managed component)
& 
-- some monitoring cycles can be redundant if scheduler is not properly tuned
\\ \hline

\textbf{data format} & & \\ \specialrule{.1em}{.05em}{.05em}
binary &
-- space-effective
& 
-- unreadable for human
\\ \hline
character-based &
-- human readable
&
-- ineffective in terms of space
\\ \hline

\textbf{communication protocol} & & \\ \specialrule{.1em}{.05em}{.05em}
vendor specific &
-- provides specific data to a given managed component (ie. virtual machine / container)
& 
-- may cause incompatibility problem while expanding cloud-sap system to a consecutive cloud instances
\\ \hline
standard-based &
-- supports scaling across multiple cloud instances as long as they support standard
&
-- gathered data may be inadequate to a specific needs
\\ \hline
\end{tabularx}

\caption{Monitoring - summary of issues}
\label{tab:monitoring-module-issues-summary}

\end{table}

Cloud-SAP is entirely focused on further data processing, consequently decision whether to store monitoring data or not is entirely implementation dependent.

\subsubsection{Analysis}
Analyse module applies a specified mathematical model to a gathered data and yields a conclusion on top of that. There is a vast number of models that can be used for that purpose, for example \cite{LiWoZh05} enumerates:
\begin{itemize}
 \item Queue-based Performance Models
 \item Dynamic models
 \item Monotonic static models
 \item Policy based models
\end{itemize}

It is vital that Cloud-SAP implementation use predictive approach \cite{JiPeLiCh11} and track changes \cite{ZhYaWo05} in a system, giving a solid foundation for a knowledge base that can contribute to a better data analysis in future.

\subsubsection{Planning}
Planning module is responsible for scheduling actions that have to be taken to solve problems reported by an earlier analysis. This may involve:
\begin{itemize}
 \item reserving some resources (for example to deploy a new virtual machine)
 \item handling situation where multiple entities compete about the resources (for example by a prioritization)
\end{itemize}

It is possible, however, that resolving issues is not possible at a current layer. It that case, the upper-layer is responsible for handling it.

\subsubsection{Execution}
It is expected that execution module manage resource by performing various operations on resource's effector. Each layer is characterised by different actions that can be taken, for example: application server reconfiguration, vertical or horizontal scaling. 

Enforcing actions on a managed component can be done using different methods, however, all of them requires an agent installed in a managed resource. Cloud-SAP does not enforce any specific technology, leaving low-level decision to an implementation.

\subsubsection{Knowledge base}
Knowledge base is represented as a set of rules, policies shared across autonomic components.

-- TODO przyklad z ibma

\subsection{Cloud federation}
Opis najwyzej warstwy, ze ma brokery, wykorzystuje kilka chmur.


\section{Auto-scaling module}

\subsection{Application platform layer}

\subsection{Container layer}
Among the currently available monitoring systems remarkable are Cloud Watch \cite{CloudWatch}, OpenNebula (Information Manager / Virtual Machine Manager) \cite{OpenNebula} there are also open approaches such as Ganglia Monitoring System \cite{MaChCu04}.

\subsection{Stack layer}

\subsection{Cloud layer}


\section{Cloud federation}

\section{Solution discussion}

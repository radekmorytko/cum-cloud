\vspace*{0.5cm}

\begin{center}
  \LARGE Abstract
\end{center}

Cloud computing has become an attractive model for on demand provisioning of computing resources as services to end-users. It is based on the assumption that almost anything can be viewed as a service, starting from applications delivered over Internet, through hardware in the data centers and ending on computing power. That model appears to be so attractive as from the user point of view the offered resources are infinite, transparent, robust and ready to consume at any time.

Most of the times end-users do not know in advance what the demand for the service is. This creates a requirement in which their systems are auto-scalable, i.e. they support sudden spikes in demand followed by underutilization at other times. An architecture of a cloud computing system that meets these requirements has a characteristic of a multi-layered self-adaptive system, where different layers corresponds to different levels where cloud operates: starting from an application layer, through the application platform, infrastructure to end on a cloud instance.

Problem complexity raises challenges in a variety of aspects, especially in terms of providing cooperation and mutual sharing of resources that may belong to different cloud providers. Therefore, an architecture that enables seamless cooperation among cloud providers and takes into account various QoS requirements of end-users is absolutely vital. The InterCloud architecture is one of the first attempts that had been made in this direction. Having characteristic of an application platform in mind, we propose a variation of that architecture.

The main contributions of this thesis are as follows: a) reference architecture that enables aforementioned scenarios, b) implementation of that architecture c) simulation and laboratory tests.

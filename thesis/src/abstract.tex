\vspace*{0.5cm}

\begin{center}
  \LARGE Abstract
\end{center}

Cloud computing has become an attractive model for provisioning on demand
computing resources as services to end-users. It is based on the assumption
that almost anything can be viewed as a service, starting from applications
delivered over Internet, through hardware in the data centers and ending on
computing power. The model appears to be so attractive as from the user point
of view the offered resources are infinite, transparent, robust and ready to
consume at any time. What is more, at present the ease and elasticity at
which the services are available is on an unprecedented scale, allowing both
individuals or organizations and companies, regardless of their needs, set up
and deploy the most sophisticated environments and infrastructures in a
matter of minutes or hours. Additionally, most of the times end-users do not
know in advance what the demand for the service is. This creates a
requirement in which their systems are auto-scalable, i.e. they support
sudden spikes in demand followed by underutilization at other times.

In order to ensure these growing and demanding aspects of Quality of Service
it is a common case when cloud providers cooperate with one-another and
mutually share their resources.

Therefore it is absolutely vital an architecture that enables seamless
cooperation among cloud providers and takes into account various QoS
requirements of end-users be developed.

The InterCloud architecture is one of the first attempts that had been made
in this direction. However, until now there are no known implementations of
it and all the tests were executed on a simulator.

The main goal of this thesis is to propose and implement our own architecture
which is based of the aforementioned work. Additionally, we want to use quite
new virtualization technology, linux containers, as a foundation of
provisioning resources of data centers.

The thesis is organized as follows \ldots


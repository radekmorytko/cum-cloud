\newglossaryentry{auto-scaling}{name={auto-scaling},  description={System ability to dynamically scale, i.e. add additional resources}}
\newglossaryentry{self-adaptation}{name={self-adaptation},  description={Ability of a system to adapt its behaviour in order to react towards the environment dynamic}}
\newglossaryentry{cloud federation}{name={cloud federation},  description={A concept that comprises services from different
providers aggregated in a single pool supporting three basic interoperability features - resource migration, resource
redundancy and combination of complementary resources
resp. services}}
\newglossaryentry{InterCloud}{name={InterCloud},  description={An example of a federated cloud computing environment architecture}}
\newglossaryentry{PaaS}{name={PaaS},  description={Service that enables deployment onto the cloud infrastructure consumer-created or acquired applications created using programming languages, libraries, services, and tools supported by the provider.}}
\newglossaryentry{IaaS}{name={IaaS},  description={Service that enables provisioning of processing, storage, networks, and other fundamental computing resources where the consumer is able to deploy and run arbitrary software, which can include operating 
systems and applications}}
\newglossaryentry{QoS}{name={QoS},  description={Totality of characteristics of a service that bear on its ability to satisfy stated and implied needs of the user of the service.}}
\newglossaryentry{SLA}{name={SLA},  description={A document which defines the relationship between two parties: the provider and the recipient. It should
\begin{inparaenum}[1)]
 \item Identify and define the customer’s needs
 \item Provide a framework for understanding
\end{inparaenum}
}}
\newglossaryentry{policy}{name={policy},  description={A principle or protocol to guide decisions and achieve rational outcomes.}}





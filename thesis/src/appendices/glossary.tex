\newglossaryentry{auto-scaling}{name={auto-scaling},  description={System ability to dynamically scale, i.e. add additional resources}}
\newglossaryentry{self-adaptation}{name={self-adaptation},  description={Ability of a system to adapt its behaviour in order to react towards the environment dynamic}}
\newglossaryentry{cloud federation}{name={cloud federation},  description={A concept that comprises services from different
providers aggregated in a single pool supporting three basic interoperability features - resource migration, resource redundancy and combination of complementary resources resp. services}}
\newglossaryentry{InterCloud}{name={InterCloud},  description={An example of a federated cloud computing environment architecture}}
\newglossaryentry{PaaS}{name={Platform-as-a-Service (PaaS)},  description={A service that enables deployment onto the cloud infrastructure consumer-created or acquired applications created using programming languages, libraries, services, and tools supported by the provider.}}
\newglossaryentry{IaaS}{name={Infrastructure-as-a-Service (IaaS)},  description={A service that enables provisioning of processing, storage, networks, and other fundamental computing resources where the consumer is able to deploy and run arbitrary software, which can include operating
systems and applications}}
\newglossaryentry{QoS}{name={Quality-of-Service (QoS)},  description={Totality of characteristics of a service that bear on its ability to satisfy stated and implied needs of the user of the service.}}
\newglossaryentry{SLA}{name={Service Level Agreement (SLA)},  description={A document which defines the relationship between two parties: the provider and the recipient. It should
\begin{inparaenum}[1)]
 \item Identify and define the customer’s needs
 \item Provide a framework for understanding
\end{inparaenum}
}}
\newglossaryentry{policy}{name={policy},  description={A principle or protocol to guide decisions and achieve rational outcomes.}}



\newglossaryentry{adaptable system}{name={adaptable system},  description={
An adaptable system is a system that has the following attributes:
\begin{inparaenum}[1)]
\item it can monitor its state and take appropriate actions,
\item IT staff is responsible for managing performance against SLAs and
\item the human interaction should be at the minimal level.
\end{inparaenum}
}} 

\newglossaryentry{autonomic system}{name={autonomic system},  description={ A
system with the ability to manage itself and dynamically adapt to change in
accordance with business policies and objectives. Self-managing environments
can perform such activities based on situations they observe or sense in the IT
environment rather than requiring IT professionals to initiate the task.  These
environments are self-configuring, self-healing, self-optimizing, and
self-protecting.  }} 

\newglossaryentry{lightweight virtualization}{name={lightweight
virtualization}, description={ Lightweight or operating system–level
  virtualization is a server virtualization method where the kernel of an
  operating system allows for multiple isolated user-space instances, instead
  of just one. Such instances (often called containers, virtualization engines
(VE), virtual private servers (VPS) or jails) may look and feel like a real
server, from the point of view of its owner. }} 

\newglossaryentry{horizontal scalability}{name={horizontal scaling},
description={ The ability to connect multiple hardware or software entities,
such as virtual machines, so that they work as a single logical unit.  }} 

\newglossaryentry{vertical scalability}{name={vertical scaling}, description={
The ability to increase the capacity of existing hardware or software by adding
resources - for example, adding processing power to a server to make it faster
}}

\newglossaryentry{hybrid cloud}{name={hybrid cloud},  description={
The cloud infrastructure is a composition of two or more distinct cloud
infrastructures (private, community, or public) that remain unique entities,
but are bound together by standardized or proprietary technology that enables
data and application portability (e.g., cloud bursting for load balancing
between clouds)
}} 


\newglossaryentry{cloud bursting}{name={cloud bursting},  description={
A deployment model which enables seamless resource infrastructure scaling for
a cloud customer which has its own private cloud. It relies on the concept of
using resources provided by an external provider for a given time interval 
when necessary, e.g. in case of a sudden spike in demand of a web application
}} 

